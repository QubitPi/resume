\documentclass[10pt,a4paper,ragged2e,withhyper]{altacv}

\geometry{left=1.25cm,right=1.25cm,top=1.5cm,bottom=1.5cm,columnsep=1.2cm}
\usepackage{paracol}

\setmainfont{Poppins}
\usepackage[fallback]{xeCJK}[2011/05/01 v2.3.19]
%\setCJKmainfont{Bauhaus ITC}
\setCJKmainfont{Songti SC}

\definecolor{VividPurple}{HTML}{3E0097}
\definecolor{SlateGrey}{HTML}{2E2E2E}
\definecolor{LightGrey}{HTML}{666666}
\colorlet{heading}{VividPurple}
\colorlet{headingrule}{VividPurple}
\colorlet{accent}{VividPurple}
\colorlet{emphasis}{SlateGrey}
\colorlet{body}{LightGrey}

\renewcommand{\cvItemMarker}{{\small\textbullet}}
\renewcommand{\cvRatingMarker}{\faCircle}

\hypersetup{colorlinks = true, linkcolor = blue, urlcolor  = blue, citecolor = blue, anchorcolor = blue}
\usepackage{worldflags}

\begin{document}
\name{刘珈奇}
\tagline{专业打造现代化、高质量、高效率的 C 端 IT 科技团队}
\photoR{2.5cm}{me}
\personalinfo{
    \github{QubitPi}
    \phone{139-2343-3392}
    \mailaddress{jack20220723@gmail.com}
    %\location{现居:深圳}
    \jobLocation{求职地意向:深圳}
}

\makecvheader

\AtBeginEnvironment{itemize}{\small}
\columnratio{0.6}
\begin{paracol}{2}

\cvsection{履历概述}
    \cvevent{创始人、技术总监}{深圳市派昂科技有限公司}{2022/9 - 至今}{中国,深圳市,福田区}
    \begin{itemize}
        \item 努力实现企业利润增长:与湖南知名口腔连锁机构达成合作意向,开发智能语音病例管理系统,大大提高医生的工作效率和医患沟通质量,并在其中设计软件架构和数据流水线
        \item 注重人才培养:在一个团队规模 8 人,研发人员 5 人的团队中,定期开展技术培训,并对不同技术人员开展针对性培养
        \item 践行“语言创造价值”的战略:研发利用知识图谱技术,帮助多语言学习者高效学习词汇的 \href{https://wilhelmlang.com/}{wilhelmlang.com}
        \item 不断推进科技创新实践:探索和研发自然语言处理和知识图谱相结合的知识管理类软件 \href{https://nexusgraph.com/}{nexusgraph.com}
        \item 采用敏捷项目管理,构建开放、自由、自律的团队文化
    \end{itemize}

    \divider

    \cvevent{资深软件开发工程师}{北京海致星途科技(深圳)有限公司}{2020/10 - 2022/05}{中国,深圳市,南山区}
    \begin{itemize}
        \item 参与开发知识图谱平台核心功能,并努力推进团队的技术标准化发展,例如帮助后端团队规范话软件开发流程,
              例如代码规范,引入自动化测试,和制定代码审核机制
        \item 帮助产品团队探索全自动化的软件运维管理平台,利用 Rancher/KubeSphere 虚拟化平台完善团队的 CI/CD 基础设施
        \item 负责开发知识图谱平台难度高,业务复杂的资金追踪功能,利用动态知识图谱展示的方式,帮助金融监管部门和银行机构追踪资金流向,精准打击金融犯罪活动
    \end{itemize}

    \divider

    \cvevent{高级软件开发工程师}{腾讯科技(深圳)有限公司}{2020/04 - 2020/07}{中国,深圳市,南山区}
    \begin{itemize}
        \item 机器学习平台后台维护开发,实现平台运营数据报表系统,通过图表的形式帮助领导迅速直观地评估机器学习平台的价值
        \item 工作高质量完成,顺利转正
        \item 对自己的职业规划有了更加正确的认识,意识到自己更加适合去一个什么样的团队
    \end{itemize}

    \divider

    \cvevent{Software Engineer}{Yahoo!}{2016/02 -- 2020/02}{硅谷, 加利福尼亚州, 美国}

    \switchcolumn

    \cvsection{目标明确}
    \begin{quote}
    全心全意,为贵企创造科技价值
    \end{quote}

    \cvsection{履历过硬}

    \cvachievement{\faYahoo}{美国 Yahoo Inc.}{Software Engineer, 全球广告大数据业务}

    \divider

    \cvachievement{\faQq}{腾讯科技(深圳)有限公司}{高级软件开发工程师,机器学习数据中台}

    \divider

    \cvachievement{\faBuilding}{深圳市派昂科技有限公司}{自主创业,语言学习 \& 知识图谱}

    \cvsection{深耕技术}

    {\LaTeXraggedright
    \cvtag{\href{https://github.com/Qubitpi}{热衷开源}}
    \cvtag{\href{https://github.com/Qubitpi\#backend-dev}{后端开发}}
    \cvtag{数据流水线}
    \cvtag{\href{https://github.com/Qubitpi\#devops}{运维开发}}
    \cvtag{\href{https://nlp.qubitpi.org/}{自然语言处理}}
    \cvtag{\href{https://github.com/Qubitpi\#frontend-dev}{前端开发}}
    \par}

    \divider\smallskip

    \cvtag{项目管理}
    \cvtag{团队管理}

    \cvsection{学历优异}

    \cvevent{计算机科学(硕士)}{University of Illinois at Urbana-Champaign(\href{https://www.usnews.com/best-graduate-schools/top-science-schools/computer-science-rankings}{美国计算机专业 TOP 5})}{2014/08 -- 2015/12}{美国,伊利诺伊州}{Big Data \& Cloud Computing, GPA: \href{https://github.com/QubitPi/resume/blob/master/(grad%20%2B%20undergrad)%20official-transcript.pdf}{\textbf{3.66/4.00}}, \href{https://github.com/QubitPi/resume/blob/master/%E5%AD%A6%E5%8E%86%E8%AE%A4%E8%AF%81-%E7%A1%95%E5%A3%AB.pdf}{\worldflag[width=0.3cm, framewidth=0mm]{CN} 教育部学历认证}}

    \divider

    \cvevent{工程物理(本科)}{University of Illinois at Urbana-Champaign(\href{https://www.usnews.com/best-colleges/rankings/national-universities/top-public?_sort=rank&_sortDirection=asc}{美国公立大学 TOP 10})}{2010/08 -- 2014/05}{美国,伊利诺伊州}{Computational Quantum Mechanics, GPA: \href{https://github.com/QubitPi/resume/blob/master/(grad%20%2B%20undergrad)%20official-transcript.pdf}{\textbf{3.63/4.00}}, \href{https://github.com/QubitPi/resume/blob/master/%E5%AD%A6%E5%8E%86%E8%AE%A4%E8%AF%81-%E6%9C%AC%E7%A7%91.pdf}{\worldflag[width=0.3cm, framewidth=0mm]{CN} 教育部学历认证}}

\end{paracol}

    \begin{itemize}
    \item 软件工程师 (2018/10 - 2020/02)
    \begin{itemize}
        \item 负责公司在线广告业务某独立数据的收集和存储项目,服务于数据部门所有下游团队
        \item 该项目使用开源技术 (Jersey, Jetty, MySQL, Hibernate ORM) 成功地为公司节省了高昂的第三方开支    (Oracle, Microstrategy)
        \item 该项目提升了公司 ETL 技术的数据涵盖范围,将可自动化上传的数据扩展到了非机器数据源
        \item 自主设计了整个软件的架构,包括前端和中间件 Webservice
        \item 独立开发了架构中的数据 API 接口,该接口能够在短时间之内迅速开发并收集新的商业数据类型
        \item 与其他 2 名工程师合作开发网页前端
        \item 负责项目的所有软件部署和维护工作,包括用 CHEF 设计跟实现软件的全自动化部署 (CI/CD),和 PE 合作部署 MySQL 数据库方案,以及数据结构与产品需求的同步
        \item 迅速实现产品经理的需求
    \end{itemize}
    \item 中级软件工程师 (2017/02 - 2018/10)
    \begin{itemize}
        \item 负责开发并维护团队新一代先后基于 MySQL 和 HBase 的公司广告元数据查找系统,支持1个 TB 以上的数据快速查找
        \item 参与开发 PB 级别利润数据计算与搜索的 Apache Druid 拓展模块
        \item 参与开发高效的中间件 Webservice 利润数据访问 API 接口,提供公司全球的利润数据和分析报表
        \item 不断与产品经理沟通并提升数据后台,以提供更好的数据访问体验
        \item 迅速将 TB 级别的数据从雅虎老数据系统迁移至 Hadoop HDFS 分布式存储系统,避免了整个数据系统的瘫痪
        \item 活跃于开源社区并成为了开源软件 fili.io 的核心开发者,参与开发了多项大的开发任务,并主动设计了软件主页 - fili.io
        \item 主动要求完成职责之外的事情,如帮助团队的 ETL 小组开发数据批量处理脚本
        \item 不断提升团队的代码,主动提议重构团队的元数据维护系统,提升系统自动化,并使代码更加容易维护
    \end{itemize}
    \item 中级软件工程师 (2016/02 - 2017/02)
    \begin{itemize}
        \item 在雅虎的 Flurry 移动分析平台团队中负责中间件 Webservice 开发
        \item 参与 Flurry 数据 API 的维护和相关开源软件 Elide(elide.io) 的开发
        \item 处理用户使用 API 时遇到的各种问题
        \item 参与新业务数据 API 的设计和开发,帮助移动 app 开发者分析 app的报错数据
        \item 维护、提升 Flurry 数据安全 API 接口并修复日常 bug
        \item 参与软件运维工作,开发所有软件的运行监控脚本,帮助团队成员更好地了解软件运行状况,例如用户访问处理时间,CPU 使用率,服务器负载,以及性能数据等
    \end{itemize}
    \end{itemize}

    \divider

    \cvevent{Software Enginner Intern}{Grainger Inc. (世界 500 强)}{2014/04 - 2014/08}{香槟市, 伊利诺伊州, 美国}
    \begin{itemize}
    \item 成功开发并部署了公司物联网传感新技术,采集发电机(中央空调等)的运行情况数据,将数据可视化呈现给用户;与客户达成了 2000 个产品的交易合同,\href{https://github.com/QubitPi/resume/blob/master/recommendation.png?raw=true}{获得部门主管的高度认可}
    \item 与另一位实习生合作用 Node.js 开发数据可视化前端,使用户可以查看发电机每秒的运行数据,帮助用户更好地预测机器故障
    \item 展示出了明显的主人翁意识,主动承接最复杂的开发功能,并指引其它实习生的完成工作
    \end{itemize}

\end{document}
